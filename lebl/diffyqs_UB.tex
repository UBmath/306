\documentclass[12pt]{book}
%\usepackage{pdf14}
%Paper saving
%\documentclass[12pt,openany]{book}
%\documentclass[10pt,openany]{book}
%\documentclass[8pt,openany]{extbook}

%FIXME: for PRINT run for lulu or kdp, search for %PRINT  (also
%search diffyqssetup.sty)


%See text for 
%FIXMEevillayouthack
% for evil layout hacks

\usepackage{diffyqssetupUB}

% Trying this to try to solve first-page-of-chapter numbering placement. JR 8/11/19
%\usepackage{fancyhdr}
%\pagestyle{fancy}
%\lhead{}
%\chead{}
%\rhead{}
%\lfoot{}
%\cfoot{\thepage}
%\rfoot{}

% Useful for making up drafts
%\usepackage{draftwatermark}
%\SetWatermarkText{Draft of v6.0 as of \today. May still change!}
%\SetWatermarkAngle{90}
%\SetWatermarkHorCenter{0.5in}
%\SetWatermarkColor[gray]{0.7}
%\SetWatermarkScale{0.18}

% This is useful for debugging
%\overfullrule=2cm

\author{Ji\v{r}\'i Lebl}

\title{Differential Equations}
% Set up our index
\makeindex

%mbx <!--BEFORE book-->

%mbxdocinfo    <website>
%mbxdocinfo        <title>www.jirka.org/diffyqs/</title>
%mbxdocinfo        <url>https://www.jirka.org/diffyqs/</url>
%mbxdocinfo    </website>
%mbxdocinfo
%mbxdocinfo    <brandlogo source="logo.png" url="https://www.jirka.org/diffyqs/" />
%mbxdocinfo
%mbxdocinfo    <rename element="inlineexercise">Exercise</rename>
%mbxdocinfo    <rename element="divisionalexercise">Exercise</rename>
%mbxdocinfo
%mbxdocinfo    <latex-preamble>
%mbxdocinfo        <package>cancel</package>
%mbxdocinfo    </latex-preamble>
%mbxdocinfo
%mbxdocinfo    <initialism>DIFFYQS</initialism>


%mbxmacro \newcommand{\nicefrac}[2]{{{}^{#1}}\!/\!{{}_{#2}}}
%mbxmacro \newcommand{\unitfrac}[3][\!\!]{#1 \,\, {{}^{#2}}\!/\!{{}_{#3}}}
%mbxmacro \newcommand{\unit}[2][\!\!]{#1 \,\, #2}
%mbxmacro \newcommand{\noalign}[1]{}
%mbxmacro \newcommand{\qed}{\qquad \Box}
%mbxmacro \newcommand{\mybxbg}[1]{\boxed{#1}}
%mbxmacro \newcommand{\mybxsm}[1]{\boxed{#1}}

\begin{document}

%mbx <title>Notes on Diffy Qs</title>
%mbx <subtitle>Differential Equations for Engineers</subtitle>
%mbx
%mbx <frontmatter>
%mbx   <titlepage>
%mbx
%mbx     <author>
%mbx       <personname>Jiří Lebl</personname>
%mbx       <department>Department of Mathematics</department>
%mbx       <institution>Oklahoma State University</institution>
%mbx       <email>jiri.lebl@gmail.com</email>
%mbx     </author>
%mbx
%mbx     <date><today /></date>
%mbx
%mbx   </titlepage>
%mbx
%mbx   <colophon>
%mbx
%mbx     <copyright>
%mbx       <year>2008<ndash />2020</year>
%mbx       <holder>Jiří Lebl</holder>
%mbx       <minilicense>Creative Commons Attribution-Non-commercial-Share Alike 4.0 International License and the Creative Commons Attribution-Share Alike 4.0 International License</minilicense>
%mbx       <shortlicense>
%mbx         This work is dual licensed under the Creative Commons Attribution-Non-commercial-Share Alike 4.0 International License and
%mbx         the Creative Commons Attribution-Share Alike 4.0 International License.  To view a copy of these licenses, visit
%mbx         <url href="https://creativecommons.org/licenses/by-nc-sa/4.0/">https://creativecommons.org/licenses/by-nc-sa/4.0/</url> or
%mbx         <url href="https://creativecommons.org/licenses/by-sa/4.0/">https://creativecommons.org/licenses/by-sa/4.0/</url>
%mbx         or send a letter to Creative Commons PO Box 1866, Mountain View, CA 94042, USA.
%mbx       </shortlicense>
%mbx     </copyright>
%mbx     <p>
%mbx        Version 6.1pre
%mbx     </p>
%mbx     <p>
%mbx       You can use, print, duplicate, share this book as much as you want.  You can
%mbx       base your own notes on it and reuse parts if you keep the license the
%mbx       same.  You can assume the license is either the CC-BY-NC-SA or CC-BY-SA,
%mbx       whichever is compatible with what you wish to do, your derivative works must
%mbx       use at least one of the licenses.
%mbx       Derivative works must be prominently marked as such.
%mbx     </p>
%mbx     <p>
%mbx       During the writing of this book, 
%mbx       the author was in part supported by NSF grant DMS-0900885 and
%mbx       DMS-1362337.
%mbx     </p>
%mbx     <p>
%mbx       The date is the main identifier of version.  The major version / edition
%mbx       number is raised only if there have been substantial changes.
%mbx       Edition number started at 5, that is, version 5.0, as it was not kept track of
%mbx       before.
%mbx     </p>
%mbx     <p>
%mbx       See <url href="https://www.jirka.org/diffyqs/">https://www.jirka.org/diffyqs/</url>
%mbx       for more information
%mbx       (including contact information).
%mbx       The LaTeX source for the book is available for possible modification and customization
%mbx       at github: <url href="https://github.com/jirilebl/diffyqs">https://github.com/jirilebl/diffyqs</url>
%mbx     </p>
%mbx
%mbx   </colophon>
%mbx </frontmatter>

%mbxSTARTIGNORE

%also fix above
\newcommand{\theversion}{6.0}
\makediffytitlepage

\newpage

\vspace*{\fill}

\begin{small}
\noindent
Typeset in \LaTeX.

\bigskip

\noindent
Copyright \copyright 2019-2020 J. Faran, B. Hassard, J. Hundley, J. Javor, J. Ringland.%Ji\v{r}\'i Lebl


%PRINT
% not for lulu
%\medskip
\noindent
Amazon KDP edition\\
ISBN-13:  978-1-65671-136-6   % new one from Joe Hundley, 1/6/20 
%ISBN-13: 978-1-70623-023-6 Lebl's

%PRINT
% not for lulu
%\medskip
%\noindent
%Cover image: Arch in St.\ Louis, \copyright 2008 Ji{\v r}\'i Lebl, all rights reserved.  Cover
%image cannot be reused in derivative works.

\bigskip

%\begin{floatingfigure}{1.4in}
%\vspace{-0.05in}
\noindent
\includegraphics[width=1.38in]{figures/license}
\quad
\includegraphics[width=1.38in]{figures/license2}
%\end{floatingfigure}

\bigskip

\noindent
This work
%PRINT
% not for lulu
%(except the cover art)
is dual licensed under
the Creative Commons
Attribution-Non\-commercial-Share Alike 4.0 International License and
the Creative Commons
Attribution-Share Alike 4.0 International License.
To view a
copy of these licenses, visit
\url{https://creativecommons.org/licenses/by-nc-sa/4.0/}
or
\url{https://creativecommons.org/licenses/by-sa/4.0/}
or send a letter to
Creative Commons
PO Box 1866, Mountain View, CA 94042, USA\@.
%Creative Commons, 171 Second Street, Suite 300, San Francisco, California,
%94105, USA.

\bigskip

\noindent
It is a derivative work of version 6.0 of \emph{Notes on DiffyQs} by Ji\v{r}\'i Lebl (github: \url{https://github.com/jirilebl/diffyqs})
which is also dual licensed under CC-BY-NC-SA and CC-BY-SA.


\bigskip

\noindent
You can use, print, duplicate, share this book as much as you want.  You can
base your own notes on it and reuse parts if you keep the license the
same.  You can assume the license is either the
CC-BY-NC-SA or CC-BY-SA\@,
whichever is compatible with what you wish to do, your derivative works must
use at least one of the licenses.
Derivative works must be prominently marked as such.
%If you plan to use it commercially (sell it for more than just
%duplicating cost), then you need to contact me and we will work something out.
%If you are printing a course pack for your students, then it is fine if the 
%duplication service is charging a fee for printing and selling the printed
%copy.  I consider that duplicating cost.

\bigskip

\noindent
During the writing of the source work, its 
author was in part supported by NSF grant DMS-0900885 and
DMS-1362337.

\bigskip

\noindent
The date is the main identifier of version.  The major version / edition
number is raised only if there have been substantial changes.
%, if only
%very minor changes or fixes are done only the minor version is raised.
Edition
number started at 5, that is, version 5.0, as it was not kept track of
before.
%The Createspace edition ISBN number identifies the major version, and is not
%changed for minor updates fixing errata.
%The edition given with the ISBN number is the major version.

\bigskip

\noindent
See \url{https://www.jirka.org/diffyqs/} for more information
(including contact information).

\bigskip

\noindent
The \LaTeX\ source for the source work is available
for possible modification and customization
at github: \url{https://github.com/jirilebl/diffyqs}


\noindent
The \LaTeX\ source for this derivative work is available
for possible modification and customization
at github: \url{https://github.com/UBmath/306}

\bigskip

\noindent
Cover photo courtesy of University at Buffalo | Chad Cooper.

\end{small}


\diffytableofcontents

\newpage

%mbxENDIGNORE

%mbx <!--HERE IS WHERE WE ADD MBX PREAMBLE AFTER book-->
%mbx <!--HERE IS WHERE WE ADD MBX PREAMBLE 2-->

%%%%%%%%%%%%%%%%%%%%%%%%%%%%%%%%%%%%%%%%%%%%%%%%%%%%%%%%%%%%%%%%%%%%%%%%%%%%%%

% Introduction chapter
\input ch-intro-withadds.tex

%%%%%%%%%%%%%%%%%%%%%%%%%%%%%%%%%%%%%%%%%%%%%%%%%%%%%%%%%%%%%%%%%%%%%%%%%%%%%%

% First order ODEs chapter
\input ch-first-order-ode-withadds.tex

%%%%%%%%%%%%%%%%%%%%%%%%%%%%%%%%%%%%%%%%%%%%%%%%%%%%%%%%%%%%%%%%%%%%%%%%%%%%%%

% Higher order linear ODEs chapter
\input ch-higher-order-ode-withadds.tex

%%%%%%%%%%%%%%%%%%%%%%%%%%%%%%%%%%%%%%%%%%%%%%%%%%%%%%%%%%%%%%%%%%%%%%%%%%%%%%

% Systems of ODEs chapter
\input ch-systems-withadds.tex

%mbxSTARTIGNORE
%This makes contents fit if needed
%FIXMEevillayouthack
\addextraspacetotoc
%mbxENDIGNORE


%%%%%%%%%%%%%%%%%%%%%%%%%%%%%%%%%%%%%%%%%%%%%%%%%%%%%%%%%%%%%%%%%%%%%%%%%%%%%%

% Fourier series and PDEs chapter
%\input ch-fourier-and-pde.tex

%%%%%%%%%%%%%%%%%%%%%%%%%%%%%%%%%%%%%%%%%%%%%%%%%%%%%%%%%%%%%%%%%%%%%%%%%%%%%%

% Eigenvalue problems chapter
%\input ch-eigenvalue-probs.tex
%%%%%%%%%%%%%%%%%%%%%%%%%%%%%%%%%%%%%%%%%%%%%%%%%%%%%%%%%%%%%%%%%%%%%%%%%%%%%%

\setcounter{chapter}{5}
% The Laplace transform chapter
\input ch-laplace-withadds.tex

%%%%%%%%%%%%%%%%%%%%%%%%%%%%%%%%%%%%%%%%%%%%%%%%%%%%%%%%%%%%%%%%%%%%%%%%%%%%%%

% Power series methods chapter
\input ch-power-ser-withadds.tex

%%%%%%%%%%%%%%%%%%%%%%%%%%%%%%%%%%%%%%%%%%%%%%%%%%%%%%%%%%%%%%%%%%%%%%%%%%%%%%

% Nonlinear systems chapter
\input ch-nonlin-systems-withadds.tex

%%%%%%%%%%%%%%%%%%%%%%%%%%%%%%%%%%%%%%%%%%%%%%%%%%%%%%%%%%%%%%%%%%%%%%%%%%%%%%
\appendix
%%%%%%%%%%%%%%%%%%%%%%%%%%%%%%%%%%%%%%%%%%%%%%%%%%%%%%%%%%%%%%%%%%%%%%%%%%%%%%

% mbx wants appendices to be in backmatter it seems

%mbx <backmatter>

%%%%%%%%%%%%%%%%%%%%%%%%%%%%%%%%%%%%%%%%%%%%%%%%%%%%%%%%%%%%%%%%%%%%%%%%%%%%%%

% Linear algebra appendix
%\input ap-linear-algebra.tex

%%%%%%%%%%%%%%%%%%%%%%%%%%%%%%%%%%%%%%%%%%%%%%%%%%%%%%%%%%%%%%%%%%%%%%%%%%%%%%

% Nonlinear systems chapter
\input ap-laplace-list.tex

%%%%%%%%%%%%%%%%%%%%%%%%%%%%%%%%%%%%%%%%%%%%%%%%%%%%%%%%%%%%%%%%%%%%%%%%%%%%%%
%%%%%%%%%%%%%%%%%%%%%%%%%%%%%%%%%%%%%%%%%%%%%%%%%%%%%%%%%%%%%%%%%%%%%%%%%%%%%%
%%%%%%%%%%%%%%%%%%%%%%%%%%%%%%%%%%%%%%%%%%%%%%%%%%%%%%%%%%%%%%%%%%%%%%%%%%%%%%

%must be in separate "paragraph" (empty lines before and after)
% This closes the chapters/appedices above and starts actual backmatter
%mbxCLOSECHAPTER

%%%%%%%%%%%%%%%%%%%%%%%%%%%%%%%%%%%%%%%%%%%%%%%%%%%%%%%%%%%%%%%%%%%%%%%%%%%%%%
%%%%%%%%%%%%%%%%%%%%%%%%%%%%%%%%%%%%%%%%%%%%%%%%%%%%%%%%%%%%%%%%%%%%%%%%%%%%%%
%%%%%%%%%%%%%%%%%%%%%%%%%%%%%%%%%%%%%%%%%%%%%%%%%%%%%%%%%%%%%%%%%%%%%%%%%%%%%%

%mbxSTARTIGNORE

%This makes contents fit if needed
%FIXMEevillayouthack
%\addextraspacetotoc

\renewcommand{\bibname}{Further Reading}

\begin{thebibliography}{MM}

\addfakecontentsline{Further Reading}

\label{furtherreading:chapter}

\bibitem[BM]{BM}
 Paul W.\ Berg and James L.\ McGregor, 
 \emph{\href{https://books.google.com/books?id=EfJQAAAAMAAJ}{Elementary
Partial Differential Equations}}, 
 Holden-Day,
 San Francisco, CA\@,
 1966.

\bibitem[BD]{BD}
 William E.\ Boyce and
 Richard C.\ DiPrima,
 \emph{\href{https://books.google.com/books?id=nYWcQgAACAAJ}{Elementary
Differential Equations and Boundary Value Problems}},
 11th edition,
 John Wiley \& Sons Inc.,
 New York, NY\@, 2017.

\bibitem[EP]{EP}
 C.H.\ Edwards and D.E.\ Penney,
 \emph{\href{https://books.google.com/books?id=wuWvoAEACAAJ}{Differential
Equations and Boundary Value Problems: Computing and Modeling}},
 5th edition,
 Pearson,
 2014.

\bibitem[F]{F}
 Stanley J.\ Farlow,
 \emph{\href{https://books.google.com/books?id=_ozWAAAAMAAJ}{An Introduction
to Differential Equations and Their Applications}},
 McGraw-Hill, Inc.,
 Princeton, NJ\@,
 1994.  (Published also by Dover Publications, 2006.)

\bibitem[I]{I}
 E.L.\ Ince,
 \emph{\href{https://books.google.com/books?id=uYz-pqUD75gC}{Ordinary
Differential Equations}},
 Dover Publications, Inc.,
 New York, NY\@,
 1956.

\bibitem[T]{T}
 William F.\ Trench,
 \emph{Elementary Differential Equations with Boundary Value
Problems}. Books and Monographs. Book 9.  2013.
\url{https://digitalcommons.trinity.edu/mono/9}

\end{thebibliography}
%mbxENDIGNORE

%mbx <references xml:id="furtherreading_chapter">
%mbx   <title>Further Reading</title>
%mbx
%mbx   <biblio type="raw" xml:id="biblio-BM" tag="BM">Paul W. Berg and
%mbx     James L. McGregor, 
%mbx     <title><url href="https://books.google.com/books?id=EfJQAAAAMAAJ"
%mbx     >Elementary Partial Differential Equations</url></title>,
%mbx     Holden-Day, San Francisco, CA, 1966.</biblio>
%mbx
%mbx   <biblio type="raw" xml:id="biblio-BD" tag="BD">William E. Boyce and
%mbx     Richard C. DiPrima,
%mbx     <title><url href="https://books.google.com/books?id=SyaVDwAAQBAJ"
%mbx     >Elementary Differential Equations and Boundary Value
%mbx     Problems</url></title>,
%mbx     11th edition, John Wiley &amp; Sons Inc., New York, NY, 2017.</biblio>
%mbx
%mbx   <biblio type="raw" xml:id="biblio-EP" tag="EP">C.H. Edwards
%mbx     and D.E. Penney,
%mbx     <title><url href="https://books.google.com/books?id=wuWvoAEACAAJ"
%mbx     >Differential Equations and Boundary Value Problems: Computing and
%mbx     Modeling</url></title>,
%mbx     5th edition, Pearson, 2014.</biblio>
%mbx
%mbx   <biblio type="raw" xml:id="biblio-F" tag="F">Stanley J. Farlow,
%mbx     <title><url href="https://books.google.com/books?id=_ozWAAAAMAAJ"
%mbx     >An Introduction to Differential Equations and Their
%mbx     Applications</url></title>,
%mbx     McGraw-Hill, Inc., Princeton, NJ, 1994.  (Published also by Dover
%mbx     Publications, 2006.)</biblio>
%mbx
%mbx   <biblio type="raw" xml:id="biblio-I" tag="I">E.L. Ince,
%mbx     <title><url href="https://books.google.com/books?id=uYz-pqUD75gC"
%mbx     >Ordinary Differential Equations</url></title>,
%mbx     Dover Publications, Inc., New York, NY, 1956.</biblio>
%mbx
%mbx   <biblio type="raw" xml:id="biblio-T" tag="T">William F. Trench,
%mbx     <title>Elementary Differential Equations with Boundary Value
%mbx     Problems</title>, Books and Monographs, Book 9,  2013,
%mbx     <url>https://digitalcommons.trinity.edu/mono/9</url>.</biblio>
%mbx
%mbx </references>



%%%%%%%%%%%%%%%%%%%%%%%%%%%%%%%%%%%%%%%%%%%%%%%%%%%%%%%%%%%%%%%%%%%%%%%%%%%%%%
%%%%%%%%%%%%%%%%%%%%%%%%%%%%%%%%%%%%%%%%%%%%%%%%%%%%%%%%%%%%%%%%%%%%%%%%%%%%%%
%%%%%%%%%%%%%%%%%%%%%%%%%%%%%%%%%%%%%%%%%%%%%%%%%%%%%%%%%%%%%%%%%%%%%%%%%%%%%%

%mbxSTARTIGNORE
\printanswers
%mbxENDIGNORE

%%%%%%%%%%%%%%%%%%%%%%%%%%%%%%%%%%%%%%%%%%%%%%%%%%%%%%%%%%%%%%%%%%%%%%%%%%%%%%
%%%%%%%%%%%%%%%%%%%%%%%%%%%%%%%%%%%%%%%%%%%%%%%%%%%%%%%%%%%%%%%%%%%%%%%%%%%%%%
%%%%%%%%%%%%%%%%%%%%%%%%%%%%%%%%%%%%%%%%%%%%%%%%%%%%%%%%%%%%%%%%%%%%%%%%%%%%%%

%mbxSTARTIGNORE
\diffyindex
%mbxENDIGNORE

%mbx   <index>
%mbx     <title>Index</title>
%mbx     <index-list />
%mbx   </index>

%%%%%%%%%%%%%%%%%%%%%%%%%%%%%%%%%%%%%%%%%%%%%%%%%%%%%%%%%%%%%%%%%%%%%%%%%%%%%%
%%%%%%%%%%%%%%%%%%%%%%%%%%%%%%%%%%%%%%%%%%%%%%%%%%%%%%%%%%%%%%%%%%%%%%%%%%%%%%
%%%%%%%%%%%%%%%%%%%%%%%%%%%%%%%%%%%%%%%%%%%%%%%%%%%%%%%%%%%%%%%%%%%%%%%%%%%%%%

%mbx </backmatter>

\end{document}

